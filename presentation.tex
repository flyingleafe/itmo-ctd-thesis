%%%%%%%%%%%%%%%%%%%%%%%%%%%%%%%%%%%%%%%%%
% Beamer Presentation
% LaTeX Template
% Version 1.0 (10/11/12)
%
% This template has been downloaded from:
% http://www.LaTeXTemplates.com
%
% License:
% CC BY-NC-SA 3.0 (http://creativecommons.org/licenses/by-nc-sa/3.0/)
%
%%%%%%%%%%%%%%%%%%%%%%%%%%%%%%%%%%%%%%%%%

%----------------------------------------------------------------------------------------
%	PACKAGES AND THEMES
%----------------------------------------------------------------------------------------

\documentclass{beamer}

\mode<presentation> {

% The Beamer class comes with a number of default slide themes
% which change the colors and layouts of slides. Below this is a list
% of all the themes, uncomment each in turn to see what they look like.

%\usetheme{default}
%\usetheme{AnnArbor}
%\usetheme{Antibes}
%\usetheme{Bergen}
%\usetheme{Berkeley}
%\usetheme{Berlin}
%\usetheme{Boadilla}
%\usetheme{CambridgeUS}
%\usetheme{Copenhagen}
%\usetheme{Darmstadt}
%\usetheme{Dresden}
%\usetheme{Frankfurt}
%\usetheme{Goettingen}
%\usetheme{Hannover}
%\usetheme{Ilmenau}
%\usetheme{JuanLesPins}
%\usetheme{Luebeck}
\usetheme{Madrid}
%\usetheme{Malmoe}
%\usetheme{Marburg}
%\usetheme{Montpellier}
%\usetheme{PaloAlto}
%\usetheme{Pittsburgh}
%\usetheme{Rochester}
%\usetheme{Singapore}
%\usetheme{Szeged}
%\usetheme{Warsaw}

% As well as themes, the Beamer class has a number of color themes
% for any slide theme. Uncomment each of these in turn to see how it
% changes the colors of your current slide theme.

%\usecolortheme{albatross}
%\usecolortheme{beaver}
%\usecolortheme{beetle}
%\usecolortheme{crane}
%\usecolortheme{dolphin}
%\usecolortheme{dove}
%\usecolortheme{fly}
%\usecolortheme{lily}
%\usecolortheme{orchid}
%\usecolortheme{rose}
%\usecolortheme{seagull}
%\usecolortheme{seahorse}
%\usecolortheme{whale}
%\usecolortheme{wolverine}

%\setbeamertemplate{footline} % To remove the footer line in all slides uncomment this line
%\setbeamertemplate{footline}[page number] % To replace the footer line in all slides with a simple slide count uncomment this line

%\setbeamertemplate{navigation symbols}{} % To remove the navigation symbols from the bottom of all slides uncomment this line
}

\usepackage[T2A]{fontenc}
\usepackage[utf8]{inputenc}
\usepackage[english,russian]{babel}
\usepackage[style=authoryear]{biblatex} 
\usepackage{graphicx} % Allows including images
\usepackage{booktabs} % Allows the use of \toprule, \midrule and \bottomrule in tables

%----------------------------------------------------------------------------------------
%	TITLE PAGE
%----------------------------------------------------------------------------------------

% The short title appears at the bottom of every slide, the full title is only on the title page
\title[DQN-routing]{Глубокие самообучающиеся агенты в мультиагентной системе маршрутизации} 

\author[Дмитрий Мухутдинов, М3438]{
    Мухутдинов Дмитрий, группа M3438 \\
    Научный руководитель: Фильченков А. А., к.ф-м.н., доцент кафедры КТ \\
    Рецензент: Тарасов В. Б., к.т.н., МГТУ им. Баумана
}% Your name
\institute[ИТМО] % Your institution as it will appear on the bottom of every slide, may be shorthand to save space
{
    Кафедра Компьютерных Технологий \\
    Факультет Информационных Технологий и Программирования \\
    Университет ИТМО, Санкт-Петербург
}
\date{\today} % Date, can be changed to a custom date

\begin{document}

\frame{\titlepage}

%----------------------------------------------------------------------------------------
%	PRESENTATION SLIDES
%----------------------------------------------------------------------------------------

\section{First Section} 

\subsection{Subsection Example} % A subsection can be created just before a set of slides with a common theme to further break down your presentation into chunks

\begin{frame}
  \frametitle{Задача маршрутизации}
  \begin{itemize}
  \item Сетевой роутинг
  \item Транспортная логистика
  \item Управление конвейерными системами
  \item Автоматическое управление городским трафиком
  \end{itemize}
\end{frame}

%------------------------------------------------

\begin{frame}
  \frametitle{Существующие решения}
  \begin{itemize}
  \item Link-state
    \begin{itemize}
    \item Open Shortest Path First (OSPF)
    \item IS-IS
    \end{itemize}
  \item Distance-vector
    \begin{itemize}
    \item RIP
    \item IGRP
    \end{itemize}
  \item Прочие
    \begin{itemize}
    \item AntNet
    \item ...
    \end{itemize}
  \end{itemize}
\end{frame}

%------------------------------------------------

\begin{frame}
  \frametitle{Проблемы}
  \begin{itemize}
  \item Примерно все алгоритмы маршрутизации заточены под компьютерные сети
  \item В других задачах существуют свои, более сложные условия
    \begin{itemize}
    \item Скорую нужно пропустить сквозь пробку, а обычный автомобиль --- нет
    \item Чемоданы бизнес-класса хочется доставить первыми
    \item ...
    \end{itemize}
  \end{itemize}
\end{frame}

%------------------------------------------------

\begin{frame}
  \frametitle{Задача}
  \center{Построить алгоритм, способный адаптироваться под гетерогенные условия}
\end{frame}

%------------------------------------------------

\begin{frame}
  \frametitle{Идея}
  \begin{itemize}
  \item Обучение с подкреплением
  \item Нейросети в качестве обучающихся агентов
  \item Q-routing (Boyan \& Littman, 1994)
  \end{itemize}
\end{frame}

%------------------------------------------------

\end{document} 
